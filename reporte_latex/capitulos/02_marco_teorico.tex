\chapter{Marco Teórico}

\section{Evaluación Educativa}
La evaluación educativa es un proceso sistemático y riguroso que permite obtener evidencias sobre el aprendizaje y desarrollo de los estudiantes. En México, el Plan Nacional para la Evaluación de los Aprendizajes (PLANEA) representa uno de los esfuerzos más importantes para evaluar de manera estandarizada la calidad educativa a nivel nacional.

\subsection{Importancia de la Evaluación Estandarizada}
Las evaluaciones estandarizadas como PLANEA cumplen múltiples funciones:

\begin{itemize}
    \item \textbf{Diagnóstico}: Permiten identificar fortalezas y áreas de oportunidad en el sistema educativo.
    \item \textbf{Monitoreo}: Facilitan el seguimiento de tendencias y evolución del logro educativo a través del tiempo.
    \item \textbf{Rendición de cuentas}: Promueven la transparencia y responsabilidad en el uso de recursos públicos destinados a la educación.
    \item \textbf{Comparabilidad}: Posibilitan la comparación entre diferentes entidades, tipos de escuela y periodos.
\end{itemize}

\subsection{Niveles de Logro en PLANEA}
PLANEA establece cuatro niveles de logro que representan el dominio de los aprendizajes evaluados:

\begin{table}[h]
\centering
\begin{tabular}{|c|p{12cm}|}
\hline
\textbf{Nivel} & \textbf{Descripción} \\
\hline
I & Los estudiantes tienen un conocimiento insuficiente de los aprendizajes clave incluidos en el currículo. Esto les impide seguir progresando satisfactoriamente en la asignatura. \\
\hline
II & Los estudiantes tienen un conocimiento apenas indispensable de los aprendizajes clave incluidos en el currículo. Esto les permite seguir progresando, aunque con dificultades, en la asignatura. \\
\hline
III & Los estudiantes tienen un conocimiento satisfactorio de los aprendizajes clave del currículo. Esto les permite seguir progresando adecuadamente en la asignatura. \\
\hline
IV & Los estudiantes tienen un conocimiento sobresaliente de los aprendizajes clave del currículo. Esto refleja un logro académico avanzado en la asignatura evaluada. \\
\hline
\end{tabular}
\caption{Descripción de los niveles de logro en PLANEA}
\label{tabla:niveles}
\end{table}

\section{Visualización de Datos Educativos}

\subsection{Principios de Visualización Efectiva}
La visualización de datos educativos debe seguir principios fundamentales para garantizar su efectividad:

\begin{itemize}
    \item \textbf{Claridad}: Representaciones visuales sencillas y fáciles de interpretar.
    \item \textbf{Precisión}: Representación exacta de los datos sin distorsiones.
    \item \textbf{Eficiencia}: Maximizar la relación entre información transmitida y espacio utilizado.
    \item \textbf{Contextualización}: Proporcionar el contexto necesario para una correcta interpretación.
    \item \textbf{Comparabilidad}: Facilitar la comparación entre diferentes categorías o periodos.
    \item \textbf{Accesibilidad}: Diseño inclusivo que considere diversas necesidades de los usuarios.
\end{itemize}

\subsection{Dashboards Interactivos}
Los dashboards interactivos representan una herramienta poderosa para el análisis de datos educativos por varias razones:

\begin{itemize}
    \item \textbf{Integración}: Reúnen múltiples visualizaciones en un solo espacio.
    \item \textbf{Interactividad}: Permiten a los usuarios explorar los datos según sus intereses específicos.
    \item \textbf{Actualización}: Pueden reflejar cambios en tiempo real o periódicos en los datos.
    \item \textbf{Personalización}: Se adaptan a diferentes necesidades y niveles de análisis.
    \item \textbf{Democratización}: Hacen accesibles datos complejos a usuarios con diferentes niveles de experiencia técnica.
\end{itemize}

\section{Métricas e Indicadores Educativos}

\subsection{Tipos de Métricas en Evaluación Educativa}
En la evaluación educativa, se utilizan diversos tipos de métricas:

\begin{itemize}
    \item \textbf{Métricas de logro}: Miden el nivel de aprendizaje alcanzado (puntajes directos, porcentajes).
    \item \textbf{Métricas de distribución}: Describen cómo se distribuyen los resultados (medias, medianas, desviaciones estándar).
    \item \textbf{Métricas de brecha}: Cuantifican diferencias entre grupos (brecha socioeconómica, de género, urbano-rural).
    \item \textbf{Métricas de tendencia}: Muestran evolución a lo largo del tiempo.
    \item \textbf{Métricas contextuales}: Relacionan resultados con factores del entorno.
\end{itemize}

\subsection{Selección de Indicadores Clave de Desempeño (KPIs)}
Los KPIs en educación deben cumplir con ciertos criterios:

\begin{itemize}
    \item \textbf{Relevancia}: Deben medir aspectos fundamentales del aprendizaje.
    \item \textbf{Validez}: Deben reflejar con precisión lo que pretenden medir.
    \item \textbf{Confiabilidad}: Deben producir resultados consistentes bajo condiciones similares.
    \item \textbf{Comparabilidad}: Deben permitir comparaciones válidas entre diferentes contextos.
    \item \textbf{Interpretabilidad}: Deben ser fácilmente comprensibles para los usuarios.
    \item \textbf{Disponibilidad}: Deben basarse en datos accesibles y actualizados.
\end{itemize}

\section{Desafíos Metodológicos}

\subsection{Cambios en Metodología de Evaluación}
Un desafío importante en el análisis longitudinal de datos educativos es manejar los cambios metodológicos entre diferentes ciclos de evaluación. Estos cambios pueden incluir:

\begin{itemize}
    \item Modificaciones en el diseño de las pruebas
    \item Cambios en las escalas de medición
    \item Variaciones en los contenidos evaluados
    \item Diferencias en los procedimientos de muestreo
    \item Actualizaciones en los criterios de clasificación
\end{itemize}

\subsection{Comparabilidad entre Diferentes Años}
Para abordar estos desafíos, existen diversas estrategias:

\begin{itemize}
    \item \textbf{Estandarización}: Convertir diferentes métricas a una escala común.
    \item \textbf{Análisis relativo}: Enfocarse en posiciones relativas más que en valores absolutos.
    \item \textbf{Transparencia}: Comunicar claramente las limitaciones de comparabilidad.
    \item \textbf{Métricas alternativas}: Desarrollar indicadores que puedan calcularse consistentemente a pesar de los cambios.
    \item \textbf{Análisis separado}: Tratar diferentes periodos como análisis independientes cuando la comparabilidad es muy limitada.
\end{itemize}

Este marco teórico fundamenta las decisiones metodológicas tomadas en el desarrollo del dashboard PLANEA, especialmente en lo referente a la selección de métricas y el diseño de visualizaciones que permitan un análisis riguroso a pesar de los cambios en la estructura de los datos entre 2015-2017 y 2022.
