\chapter{Introducción}

\section{Contexto Educativo en México}
La evaluación educativa en México ha sido un pilar fundamental para el desarrollo de políticas públicas orientadas a mejorar la calidad de la educación. El Plan Nacional para la Evaluación de los Aprendizajes (PLANEA) fue implementado en 2015 como una iniciativa del Instituto Nacional para la Evaluación de la Educación (INEE) con el objetivo de conocer la medida en que los estudiantes logran el dominio de un conjunto de aprendizajes esenciales al término de los distintos niveles de la educación obligatoria.

PLANEA evalúa principalmente dos áreas fundamentales:
\begin{itemize}
    \item \textbf{Lenguaje y Comunicación}: Comprensión lectora, expresión escrita y habilidades comunicativas.
    \item \textbf{Matemáticas}: Razonamiento matemático, resolución de problemas y pensamiento cuantitativo.
\end{itemize}

Los resultados de estas evaluaciones se categorizan en cuatro niveles de logro:
\begin{itemize}
    \item \textbf{Nivel I}: Logro insuficiente
    \item \textbf{Nivel II}: Logro apenas indispensable
    \item \textbf{Nivel III}: Logro satisfactorio
    \item \textbf{Nivel IV}: Logro sobresaliente
\end{itemize}

\section{Justificación del Dashboard}
Ante la gran cantidad de datos generados por PLANEA y la necesidad de facilitar su análisis e interpretación, se ha desarrollado un dashboard interactivo que permite visualizar, analizar y comparar los resultados educativos a nivel nacional y por entidad federativa.

Este dashboard responde a necesidades específicas de diferentes actores del sistema educativo:

\begin{itemize}
    \item \textbf{Autoridades educativas}: Para la toma de decisiones informadas y diseño de políticas públicas.
    \item \textbf{Directivos escolares}: Para identificar áreas de oportunidad y fortalezas en sus instituciones.
    \item \textbf{Investigadores}: Para realizar análisis profundos sobre tendencias y factores asociados al logro educativo.
    \item \textbf{Público general}: Para conocer de manera transparente los resultados del sistema educativo.
\end{itemize}

\section{Objetivos del Dashboard}
El dashboard PLANEA tiene como objetivos principales:

\begin{enumerate}
    \item Presentar de manera visual e interactiva los resultados de las evaluaciones PLANEA 2015-2017 y 2022.
    \item Facilitar la comparación de resultados entre entidades federativas, años y tipos de escuela.
    \item Proporcionar indicadores clave de desempeño (KPIs) que permitan una rápida evaluación del estado de la educación.
    \item Permitir análisis detallados mediante filtros y selecciones dinámicas.
    \item Garantizar la transparencia y accesibilidad de los datos educativos para todos los interesados.
    \item Adaptar las visualizaciones a los cambios metodológicos entre diferentes años de evaluación.
\end{enumerate}

\section{Estructura del Reporte}
El presente reporte está organizado en diez capítulos que documentan exhaustivamente el proceso de desarrollo, implementación y resultados del dashboard PLANEA:

\begin{enumerate}
    \item \textbf{Introducción}: Contexto, justificación y objetivos del dashboard.
    \item \textbf{Marco Teórico}: Fundamentos conceptuales de la evaluación educativa y visualización de datos.
    \item \textbf{Datos y Fuentes}: Descripción detallada de las fuentes de datos utilizadas.
    \item \textbf{Preprocesamiento de Datos}: Métodos y técnicas para la limpieza y preparación de los datos.
    \item \textbf{Selección de Métricas}: Justificación y cálculo de las métricas utilizadas.
    \item \textbf{Arquitectura del Dashboard}: Estructura y componentes tecnológicos.
    \item \textbf{Visualización de Datos}: Diseño y justificación de las visualizaciones implementadas.
    \item \textbf{Resultados y Análisis}: Hallazgos principales derivados del dashboard.
    \item \textbf{Limitaciones y Consideraciones}: Restricciones metodológicas y aspectos a considerar.
    \item \textbf{Conclusiones y Recomendaciones}: Síntesis y propuestas para el futuro.
\end{enumerate}

Los apéndices complementan el reporte con el código fuente completo, la estructura detallada de los datos y un manual de usuario para maximizar el aprovechamiento del dashboard.
