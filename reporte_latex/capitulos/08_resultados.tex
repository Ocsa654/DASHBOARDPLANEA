\chapter{Resultados y Análisis}

\section{Panorama General del Desempeño Educativo}
Los resultados obtenidos a través del dashboard PLANEA proporcionan un panorama integral del desempeño educativo en México durante los períodos 2015-2017 y 2022, revelando patrones, tendencias y desafíos significativos en el sistema educativo nacional.

\subsection{Desempeño Nacional}
A nivel nacional, los datos analizados muestran las siguientes tendencias generales:

\begin{itemize}
    \item \textbf{Niveles satisfactorios (2015-2017)}: El porcentaje de estudiantes que alcanzaron niveles satisfactorios (III y IV) se mantuvo consistentemente bajo, oscilando entre 15\% y 25\% dependiendo del año y la materia.
    
    \item \textbf{Diferencia entre materias}: Se observa una brecha persistente entre Lenguaje y Matemáticas, con esta última mostrando resultados sistemáticamente inferiores.
    
    \item \textbf{Evolución temporal}: Entre 2015 y 2017 se observaron mejoras modestas pero consistentes en ambas materias, con un incremento promedio de 1-2 puntos porcentuales por año.
    
    \item \textbf{Calificaciones 2022}: Las calificaciones promedio nacionales para 2022 reflejan la persistencia de desafíos educativos, aunque la diferente metodología limita comparaciones directas con el período anterior.
\end{itemize}

\begin{figure}[h]
    \centering
    % Incluir la imagen del gráfico nacional cuando esté disponible
    % \includegraphics[width=0.8\textwidth]{../imagenes/tendencia_nacional.png}
    \caption{Evolución del porcentaje de estudiantes en niveles satisfactorios a nivel nacional (2015-2017)}
    \label{fig:tendencia_nacional}
\end{figure}

\section{Análisis por Tipo de Escuela}
El análisis por tipo de escuela revela disparidades significativas en el sistema educativo mexicano.

\subsection{Desempeño Comparativo}
Los resultados muestran un patrón consistente en la jerarquía de desempeño entre los diferentes tipos de escuela:

\begin{itemize}
    \item \textbf{Escuelas particulares}: Consistentemente obtienen los mejores resultados, con porcentajes de niveles satisfactorios que duplican o triplican el promedio nacional.
    
    \item \textbf{Escuelas autónomas}: Ocupan el segundo lugar en desempeño, con resultados moderadamente superiores al promedio nacional.
    
    \item \textbf{Escuelas federales}: Presentan resultados cercanos al promedio nacional, con variaciones según el año y la materia.
    
    \item \textbf{Escuelas estatales}: Tienden a mostrar los resultados más bajos, especialmente en Matemáticas.
\end{itemize}

\subsection{Brechas de Desigualdad}
El análisis cuantifica las brechas educativas:

\begin{itemize}
    \item La diferencia entre escuelas particulares y estatales en porcentaje de niveles satisfactorios alcanza hasta 30 puntos porcentuales en algunos casos.
    
    \item Esta brecha se ha mantenido relativamente estable durante el período analizado, sin signos claros de reducción.
    
    \item En 2022, las calificaciones promedio muestran patrones similares de desigualdad, sugiriendo la persistencia de estos desafíos estructurales.
\end{itemize}

\begin{figure}[h]
    \centering
    % \includegraphics[width=0.8\textwidth]{../imagenes/comparativa_tipos_escuela.png}
    \caption{Comparativa de desempeño por tipo de escuela (2017)}
    \label{fig:comparativa_tipos}
\end{figure}

\section{Análisis Geográfico}
El análisis por entidad federativa revela importantes disparidades regionales en el desempeño educativo.

\subsection{Distribución Geográfica del Desempeño}
Los resultados muestran patrones geográficos consistentes:

\begin{itemize}
    \item \textbf{Entidades de alto desempeño}: Ciudad de México, Aguascalientes, Querétaro y Jalisco tienden a posicionarse consistentemente en los primeros lugares.
    
    \item \textbf{Entidades de bajo desempeño}: Chiapas, Guerrero, Tabasco y Michoacán frecuentemente aparecen en las últimas posiciones.
    
    \item \textbf{Correlación con desarrollo socioeconómico}: Se observa una correlación positiva entre el desempeño educativo y diversos indicadores de desarrollo económico y social.
\end{itemize}

\subsection{Casos de Éxito y Rezago}
El análisis detallado identifica casos particulares de interés:

\begin{itemize}
    \item \textbf{Casos de mejora destacada}: Entidades como Sonora y Sinaloa mostraron mejoras sostenidas superiores al promedio nacional durante 2015-2017.
    
    \item \textbf{Casos de estancamiento}: Algunas entidades como Oaxaca presentaron poca o nula mejora durante el período estudiado.
    
    \item \textbf{Brecha máxima}: La diferencia entre la entidad mejor y peor posicionada llega a superar los 20 puntos porcentuales en niveles satisfactorios.
\end{itemize}

\begin{figure}[h]
    \centering
    % \includegraphics[width=0.8\textwidth]{../imagenes/ranking_entidades.png}
    \caption{Ranking de entidades por porcentaje en niveles satisfactorios (2017)}
    \label{fig:ranking_entidades}
\end{figure}

\section{Análisis de Correlaciones}
El análisis de correlaciones revela relaciones significativas entre diferentes variables educativas.

\subsection{Correlaciones entre Materias}
Las correlaciones entre desempeño en diferentes materias muestran:

\begin{itemize}
    \item Una correlación positiva fuerte (r > 0.8) entre el desempeño en Lenguaje y Matemáticas, tanto en 2015-2017 como en 2022.
    
    \item Esta correlación sugiere que los factores que influyen en el aprendizaje tienden a afectar de manera similar a ambas materias.
    
    \item Sin embargo, la intensidad de esta correlación varía según el tipo de escuela, siendo más fuerte en escuelas particulares y más débil en escuelas estatales.
\end{itemize}

\subsection{Correlaciones con Variables Socioeconómicas}
El análisis de correlaciones con factores contextuales revela:

\begin{itemize}
    \item Correlación positiva moderada a fuerte entre el nivel socioeconómico de la escuela y el porcentaje de niveles satisfactorios.
    
    \item Correlación negativa entre indicadores de marginación y el desempeño educativo.
    
    \item Estas correlaciones apuntan a la persistencia de desafíos de equidad en el sistema educativo mexicano.
\end{itemize}

\begin{figure}[h]
    \centering
    % \includegraphics[width=0.8\textwidth]{../imagenes/matriz_correlacion.png}
    \caption{Matriz de correlación entre variables educativas clave (2017)}
    \label{fig:matriz_correlacion}
\end{figure}

\section{Análisis de los Datos 2022}

\subsection{Distribución de Calificaciones}
El análisis de los datos de 2022 muestra patrones distintivos:

\begin{itemize}
    \item Las distribuciones de calificaciones en ambas materias presentan una forma aproximadamente normal, con ligera asimetría negativa.
    
    \item La dispersión (desviación estándar) es mayor en Matemáticas que en Lenguaje, sugiriendo mayores disparidades en esta materia.
    
    \item Se observan diferencias significativas en las medias por tipo de escuela, replicando patrones similares a los observados en 2015-2017.
\end{itemize}

\subsection{Comparativa con Períodos Anteriores}
Aunque la comparación directa es limitada, se pueden extraer algunas conclusiones:

\begin{itemize}
    \item La jerarquía de desempeño entre tipos de escuela se mantiene consistente con el período 2015-2017.
    
    \item La distribución geográfica del desempeño muestra patrones similares, con las mismas entidades tendiendo a aparecer en extremos superiores e inferiores.
    
    \item Esto sugiere que, a pesar de los cambios metodológicos en la evaluación, las desigualdades estructurales en el sistema educativo persisten.
\end{itemize}

\begin{figure}[h]
    \centering
    % \includegraphics[width=0.8\textwidth]{../imagenes/histograma_2022.png}
    \caption{Distribución de calificaciones en Lenguaje y Matemáticas (2022)}
    \label{fig:histograma_2022}
\end{figure}

\section{Tendencias y Patrones Longitudinales}

\subsection{Evolución del Desempeño 2015-2017}
El análisis longitudinal para el período 2015-2017 revela:

\begin{itemize}
    \item Un incremento moderado pero sostenido en el porcentaje de estudiantes en niveles satisfactorios.
    
    \item La tasa de mejora fue ligeramente mayor en Matemáticas que en Lenguaje, aunque partiendo de niveles iniciales más bajos.
    
    \item Las brechas entre tipos de escuela y entre entidades se mantuvieron relativamente estables durante este período.
\end{itemize}

\subsection{Patrones Consistentes}
A lo largo de todos los períodos analizados, se observan patrones consistentes:

\begin{itemize}
    \item La persistencia de desigualdades educativas correlacionadas con factores socioeconómicos y geográficos.
    
    \item Desafíos particulares en el área de Matemáticas, con resultados sistemáticamente inferiores a los de Lenguaje.
    
    \item Mayor variabilidad en los resultados de escuelas estatales y federales comparadas con particulares, sugiriendo mayor heterogeneidad en el sector público.
\end{itemize}

\section{Implicaciones para Políticas Educativas}

\subsection{Áreas Prioritarias Identificadas}
El análisis permite identificar áreas que requieren atención prioritaria:

\begin{itemize}
    \item \textbf{Fortalecimiento de Matemáticas}: Los resultados consistentemente más bajos en esta materia señalan la necesidad de estrategias específicas.
    
    \item \textbf{Reducción de brechas}: Las persistentes disparidades entre tipos de escuela y regiones demandan políticas compensatorias.
    
    \item \textbf{Atención a entidades rezagadas}: Las entidades con desempeño consistentemente bajo requieren intervenciones focalizadas.
    
    \item \textbf{Aprovechamiento de casos exitosos}: Estudiar y potencialmente replicar estrategias de entidades que han mostrado mejoras significativas.
\end{itemize}

\subsection{Evaluación de Políticas Existentes}
Los datos permiten una evaluación preliminar de políticas educativas:

\begin{itemize}
    \item Las mejoras moderadas en 2015-2017 sugieren efectos positivos pero limitados de las políticas implementadas en ese período.
    
    \item La persistencia de brechas apunta a la insuficiencia de los mecanismos de compensación y equidad existentes.
    
    \item Los patrones observados en 2022 indican que los desafíos estructurales continúan presentes a pesar de cambios en políticas educativas.
\end{itemize}

\section{Hallazgos Clave}

Los hallazgos más significativos derivados del análisis pueden resumirse en:

\begin{enumerate}
    \item \textbf{Bajo nivel general}: El porcentaje de estudiantes que alcanzan niveles satisfactorios es consistentemente bajo a nivel nacional, indicando desafíos fundamentales en la calidad educativa.
    
    \item \textbf{Desigualdades persistentes}: Existen brechas significativas y persistentes entre tipos de escuela y regiones geográficas, reflejando y potencialmente reforzando desigualdades socioeconómicas.
    
    \item \textbf{Mejoras graduales}: Entre 2015 y 2017 se observaron mejoras modestas pero consistentes, sugiriendo un progreso lento pero positivo.
    
    \item \textbf{Desafío en Matemáticas}: El desempeño en Matemáticas es sistemáticamente inferior al de Lenguaje, señalando desafíos particulares en esta área.
    
    \item \textbf{Correlación socioeconómica}: Existe una correlación significativa entre factores socioeconómicos y desempeño educativo, subrayando la influencia del contexto en los resultados.
    
    \item \textbf{Continuidad de patrones}: Los datos de 2022, a pesar de diferencias metodológicas, muestran la persistencia de patrones similares de desigualdad y desafíos.
\end{enumerate}

Estos hallazgos proporcionan una base empírica sólida para la formulación de políticas educativas más efectivas y focalizadas, orientadas a mejorar la calidad educativa y reducir las brechas de desigualdad en el sistema educativo mexicano.
