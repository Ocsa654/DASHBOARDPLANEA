\chapter{Datos y Fuentes}

\section{Descripción de las Fuentes de Datos}
Para el desarrollo del dashboard PLANEA, se utilizaron dos fuentes de datos principales, proporcionadas por las autoridades educativas de México:

\begin{itemize}
    \item \textbf{DATOS2015-2017.xlsx}: Contiene los resultados de PLANEA para los años 2015, 2016 y 2017.
    \item \textbf{DATOS2022.xlsx}: Contiene los resultados de PLANEA para el año 2022.
\end{itemize}

Estos archivos representan conjuntos de datos oficiales con información detallada sobre el desempeño de los estudiantes en las pruebas PLANEA de Lenguaje y Comunicación y Matemáticas.

\section{Estructura de los Datos 2015-2017}

\subsection{Variables Principales}
El archivo DATOS2015-2017.xlsx contiene las siguientes variables clave:

\begin{table}[h]
\centering
\begin{tabular}{|p{4cm}|p{8cm}|}
\hline
\textbf{Variable} & \textbf{Descripción} \\
\hline
ENTIDAD & Entidad federativa (estado) \\
\hline
AÑO & Año de aplicación de la prueba (2015, 2016 o 2017) \\
\hline
TIPO DE ESCUELA & Categorización del centro educativo (AUTÓNOMAS, ESTATAL, FEDERAL, PARTICULARES) \\
\hline
\% NIVEL I & Porcentaje de estudiantes en nivel insuficiente \\
\hline
\% NIVEL II & Porcentaje de estudiantes en nivel apenas indispensable \\
\hline
\% NIVEL III & Porcentaje de estudiantes en nivel satisfactorio \\
\hline
\% NIVEL IV & Porcentaje de estudiantes en nivel sobresaliente \\
\hline
\end{tabular}
\caption{Variables principales en datos 2015-2017}
\label{tabla:variables2015}
\end{table}

\subsection{Características de los Datos}
Los datos para el período 2015-2017 presentan las siguientes características:

\begin{itemize}
    \item \textbf{Estructura consistente}: Mantienen el mismo formato y variables a lo largo de los tres años.
    \item \textbf{Desagregación}: Datos desagregados por entidad federativa, año y tipo de escuela.
    \item \textbf{Distribución por niveles}: Los resultados se presentan como distribución porcentual de estudiantes en cada nivel de logro.
    \item \textbf{Completitud}: La mayoría de las entidades y tipos de escuela cuentan con datos para los tres años.
\end{itemize}

\section{Estructura de los Datos 2022}

\subsection{Variables Principales}
El archivo DATOS2022.xlsx presenta una estructura diferente, con las siguientes variables clave:

\begin{table}[h]
\centering
\begin{tabular}{|p{4cm}|p{8cm}|}
\hline
\textbf{Variable} & \textbf{Descripción} \\
\hline
ENTIDAD & Entidad federativa (estado) \\
\hline
TIPO DE ESCUELA & Categorización del centro educativo \\
\hline
CALIF LENGUAJE & Calificación promedio en Lenguaje y Comunicación \\
\hline
CALIF MATEMATICAS & Calificación promedio en Matemáticas \\
\hline
\end{tabular}
\caption{Variables principales en datos 2022}
\label{tabla:variables2022}
\end{table}

\subsection{Diferencias con Datos 2015-2017}
Los datos de 2022 presentan importantes diferencias estructurales:

\begin{itemize}
    \item \textbf{Ausencia de niveles}: No incluyen porcentajes de estudiantes por nivel de logro (\% NIVEL I, II, III y IV).
    \item \textbf{Calificaciones directas}: Presentan calificaciones promedio en escala numérica en lugar de distribuciones porcentuales.
    \item \textbf{Mismo nivel de desagregación}: Mantienen la desagregación por entidad y tipo de escuela.
\end{itemize}

Esta diferencia estructural representa uno de los principales desafíos para el análisis comparativo entre los diferentes años, requiriendo estrategias específicas que se detallan en capítulos posteriores.

\section{Calidad y Limitaciones de los Datos}

\subsection{Evaluación de la Calidad}
Se realizó una evaluación sistemática de la calidad de los datos, considerando:

\begin{itemize}
    \item \textbf{Completitud}: Verificación de valores faltantes por entidad, año y tipo de escuela.
    \item \textbf{Consistencia}: Análisis de coherencia interna (por ejemplo, que los porcentajes sumen 100\%).
    \item \textbf{Exactitud}: Validación de valores extremos o atípicos.
    \item \textbf{Relevancia}: Evaluación de la pertinencia de las variables disponibles para los objetivos del análisis.
\end{itemize}

\subsection{Limitaciones Identificadas}
Durante este proceso, se identificaron algunas limitaciones importantes:

\begin{itemize}
    \item \textbf{Cambio metodológico}: La diferencia estructural entre los datos 2015-2017 y 2022 limita la comparabilidad directa.
    \item \textbf{Datos faltantes}: Algunas combinaciones de entidad-año-tipo de escuela no cuentan con información completa.
    \item \textbf{Nivel de granularidad}: Los datos están agregados a nivel entidad y tipo de escuela, sin permitir análisis a nivel escuela o estudiante.
    \item \textbf{Variables contextuales}: Limitada información sobre factores socioeconómicos, infraestructura u otros determinantes que podrían enriquecer el análisis.
\end{itemize}

\section{Estrategia de Carga y Gestión de Datos}

\subsection{Enfoque de Carga}
Para gestionar eficientemente los datos en el dashboard, se implementó un sistema de carga con las siguientes características:

\begin{itemize}
    \item \textbf{Control de volumen}: Implementación de un deslizador en la barra lateral que permite al usuario controlar el número máximo de filas cargadas.
    \item \textbf{Carga selectiva}: Optimización para cargar solo los datos necesarios según las selecciones del usuario.
    \item \textbf{Caché}: Implementación de mecanismos de caché para evitar recargar datos innecesariamente.
\end{itemize}

\subsection{Preparación Preliminar}
Antes de su visualización, los datos pasan por un proceso de preparación que incluye:

\begin{itemize}
    \item \textbf{Normalización de columnas}: Eliminación de acentos y estandarización de nombres de columnas para facilitar su procesamiento.
    \item \textbf{Filtrado inicial}: Exclusión de filas con valores nulos en variables críticas.
    \item \textbf{Transformación de tipos}: Conversión de variables a los tipos de datos apropiados.
    \item \textbf{Cálculo de métricas derivadas}: Generación de métricas compuestas como el porcentaje satisfactorio.
\end{itemize}

El código de esta fase de procesamiento se implementa en el archivo \texttt{utils.py}, que incluye funciones de utilidad para la normalización y transformación de los datos.

\section{Consideraciones Éticas}
En el manejo de estos datos educativos, se han considerado aspectos éticos fundamentales:

\begin{itemize}
    \item \textbf{Transparencia}: Comunicación clara sobre las fuentes, limitaciones y métodos de procesamiento.
    \item \textbf{Interpretación responsable}: Presentación de los datos con el contexto necesario para evitar interpretaciones erróneas.
    \item \textbf{Privacidad}: Trabajo exclusivo con datos agregados que no permiten la identificación de estudiantes individuales.
    \item \textbf{Accesibilidad}: Diseño orientado a democratizar el acceso a la información educativa.
\end{itemize}

Estas consideraciones guían tanto el procesamiento técnico como la presentación final de los datos en el dashboard.
